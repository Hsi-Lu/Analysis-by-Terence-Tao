\documentclass{article}
\usepackage{xeCJK} % For CJK support

\usepackage[margin=1in]{geometry} % Set all margins to 1 inch
\usepackage{parskip}% Add more space

\usepackage{amsmath} % is needed to use the math environments align, aligned, gather, gathered, multline etc.
\usepackage{amsthm} % For theorem environments
\usepackage{amsfonts} %For math fonts like \mathbb{R}


\usepackage{braket} % For \set
\usepackage{soul} %For strikethrough text

\usepackage{graphicx}
\graphicspath{ {./media/} }

\usepackage[dvipsnames]{xcolor} %loads 68 named colors (CMYK)
\usepackage{hyperref}
% Custom hyperlink colors
\hypersetup{
    colorlinks=true,
    linkcolor=RoyalBlue,    % Internal links
    filecolor=ForestGreen,  % Local files
    urlcolor=RedOrange,     % Web URLs
    citecolor=BrickRed,     % Citations
}

\newcommand{\doubleplus}{\mathbin{{+}{+}}}

\begin{document}
\section{Page 20}
Notice that we can prove easily, using Axioms 2.1, 2.2, and induction (Axiom
2.5), that the sum of two natural numbers is again a natural number (why?).

\begin{quotation}

    \textbf{Axiom 2.1} $0$ is a natural number.

    \textbf{Axiom 2.2} If $n$ is a natural number, then $n\doubleplus$ is also a natural number.

    \textbf{Axiom 2.5} (Principle of mathematical induction). Let $P(n)$ be any property per
    taining to a natural number $n$. Suppose that $P(0)$ is true, and suppose that whenever
    $P(n)$ is true, $P(n\doubleplus)$ is also true. Then $P(n)$ is true for every natural number $n$.

    \textbf{Definition 2.2.1} (Addition of natural numbers). Let m be a natural number. To add
    zero to $m$, we define $0 +m := m$.
    Now suppose inductively that we have defined how to add $n$ to $m$.
    Then we can add $n\doubleplus$ to $m$ by defining $(n\doubleplus)+m := (n +m)\doubleplus$.

\end{quotation}

\begin{proof}
    Fix any natural number $m$,
    we want to show that for any natural number $n$, $n + m$ is a natural number.

    \textbf{Base case}:  We have $0 + m = m$, which is a natural number by Axiom 2.1.

    \textbf{Inductive step}: Assume that for some natural number $n$, $n + m$ is a natural number.

    By Definition 2.2.1, we have $(n\doubleplus) + m = (n + m) \doubleplus$.
    Since $n + m$ is a natural number, by Axiom 2.2, $(n + m) \doubleplus$ is also a natural number.
    Thus, $(n\doubleplus) + m$ is a natural number.

    Therefore, by the principle of mathematical induction (Axiom 2.5),
    we conclude that for any natural number $n$, $n+m$ is again a natural number.

    Since the arbitrary choice of $m$ was made,
    we can conclude that the sum of two natural numbers is again a natural number.

\end{proof}

\section{Thoughts on Proposition 2.1.16}
If the family of functions $f_n$ is replaced with a single function
$f:\mathbb{N}\to\mathbb{N}$ in the proposition, will the meaning conveyed change
\begin{quotation}
    \textbf{Proposition 2.1.16} (Recursive definitions). Suppose for each natural
    number $n$, we have some function $f_n : \mathbb{N} \to \mathbb{N}$ from the
    natural numbers to the natural numbers. Let $c$ be a natural number. Then we
    can assign a unique natural number $a_n$ to each natural number $n$, such that
    $a_0 = c$ and $a_{n\doubleplus} = f_n(a_n)$ for each natural number $n$.
\end{quotation}

No. If replaced, when $a_n=a_m$, then $a_{n\doubleplus}=a_{m\doubleplus}$.
In the original proposition, we don't have such constraint.

\section{Page 21}
As a particular corollary of Lemma 2.2.2 and Lemma 2.2.3 we see that $n\doubleplus =n +1$(why?).

\begin{quotation}

    \textbf{Lemma 2.2.2} For any natural number $n$, $n +0 = n$.

    \textbf{Lemma 2.2.3} For any natural numbers $n$ and $m$, $n +(m\doubleplus) = (n +m)\doubleplus$.

\end{quotation}

\begin{proof}
    In Lemma 2.2.3, let $m = 0$.
    Then we have
    \[n + (0\doubleplus) = (n + 0) \doubleplus\]
    By Lemma 2.2.2, we have $n + 0 = n$.
    Thus, the left hand side becomes $n + 1$.
    The right hand side becomes $(n + 0) \doubleplus = n \doubleplus$.
\end{proof}

\section{Page 23}
When $a = 0$ we have $0\le b$ for all $b$ (why?)

If $a > b$, then $a\doubleplus  > b$ (why?).

If $a = b$, then $a\doubleplus  > b$ (why?).

\begin{proof}
    See solution to Exercise 2.2.4
\end{proof}

\section{Another handy lemma in textbook Chapter 2.2}
\textbf{Lemma}. For any natural number $a$ and $b$, $a  <b\doubleplus$ iff. $a \le b$.
\begin{proof}
    Included in solution to Exercise 2.2.5
\end{proof}

\section{Exercise 2.2.7}
Solution found from \url{https://math.stackexchange.com/a/4730660}

\textbf{Exercise 2.2.7} Let $n$ be a natural number,
and let $P(m)$ be a property pertaining to the natural numbers such that
whenever $P(m)$ is true, $P(m\doubleplus)$ is true.
Show that if $P(n)$ is true, then $P(m)$ is true for all $m \ge n$.
(This principle is sometimes referred to as the principle of induction starting from the base case n.)
\begin{quotation}
    \textbf{Definition 2.2.11} (Ordering of the natural numbers)
    Let $n$ and $m$ be natural numbers.
    We say that $n$ is greater than or equal to $m$, and write $n \ge m$ or $m \le n$,iff we
    have $n = m +a$ for some natural number $a$. We say that $n$ is strictly greater than $m$,
    and write $n > m$ or $m < n$,iff $n \ge m$ and $n= m$.

\end{quotation}

\begin{proof}
    Define $Q(m):=P(n+m)$. Note that, in particular, $Q(0)=P(n)$.

    We are looking to prove the statement "Whenever $P(m)$
    is true, $P(m\doubleplus )$ is true. If $P(n)$
    is true, then $P(m)$
    is true for $m\ge n$
    ", and we will do so via induction on $Q$.

    For the base case, we are already told we can assume $P(n)$
    is true, so $Q(0)$
    is true.

    Then for the inductive step,
    we have $Q(m)=P(n+m)\implies P((n+m)\doubleplus )=P(n+(m\doubleplus ))=Q(m\doubleplus )$
    (by  Lemma2.2.3, already quoted in above section not far away).
    But since $Q(m)\implies Q(m\doubleplus )$
    and $Q(0)$, then $Q(m)$ is true for every natural m
    (by Axiom 2.5 Principle of mathematical induction).

    Now, consider $P(m)$
    for $m\ge n$.
    By  Definition 2.2.11 (Ordering of the natural numbers), this means that $m=n+a$
    for some natural number $a$.
    That means we can write $P(m)=P(n+a)=Q(a)$,
    but by our inductive proof $Q(a)$ is true for any natural number $a$,
    therefore $P(m)$ is also true, and that completes the proof.
\end{proof}

\section{Exercise 3.2.1}
Exercise 3.2.1 Show that the universal specification axiom, Axiom 3.9, if assumed to be true,
would imply Axioms 3.3, 3.4, 3.5, 3.6,and 3.7. (If we assume that all natural numbers are objects,
we also obtain Axiom 3.8.) Thus, this axiom, if permitted, would simplify the foundations of set
theory tremendously (and can be viewed as one basis for an intuitive model of set theory known as
“naive set theory”). Unfortunately, as we have seen, Axiom 3.9 is “too good to be true”!
\begin{quotation}
    \textbf{Axiom 3.3} (Empty set).There exists a set $\emptyset$, known as the empty set,which contains
    no elements, i.e., for every object $x$ we have $x \notin \emptyset$.

    \textbf{Axiom 3.4} (Singleton sets \st{and pair sets}). If $a$ is an object, then there exists a set
    $\set{a}$ whose only element is $a$, i.e., for every object $y$,we have $y \in\set{a}$ if and only if
    $y =a$; we refer to $\set{a}$ as the singleton set whose element is $a$.

    \textbf{Axiom 3.5} (Pairwise union). Given any two sets $A$, $B$, there exists a set $A \cup B$,
    called the union of $A$ and $B$, which consists of all the elements which belong to $A$
    or $B$ or both. In other words, for any object $x$,
    \[x \in A\cup B \iff (x \in A\ \text{or}\ x \in B)\]

    \textbf{Axiom 3.6} (Axiom of specification). Let $A$ be a set, and for each $x \in  A$, let $P(x)$
    be a property pertaining to $x$ (i.e., for each $x \in  A$, $P(x)$ is either a true statement or
    a false statement). Then there exists a set, called $\set{x \in  A : P(x) is true}$ (or simply
    $\set{x \in  A : P(x)}$ for short), whose elements are precisely the elements $x$ in $A$ for
    which $P(x)$ is true. In other words, for any object $y$,
    \[y \in
        \set{x \in  A: P(x)\ \text{is true}}
        \iff(y \in  A\ \text{and}\ P(y)\ \text{is true})\]

    \textbf{Axiom 3.7} (Replacement). Let A be a set. For any object $x\in A$, and any object
    $y$, suppose we have a statement $P(x, y)$ pertaining to $x$ and $y$, such that for each
    $x\in A$ there is at most one $y$ for which $P(x,y)$ is true. Then there exists a set
    $\set{y : P(x, y)\text{ is true for some } x\in A}$, such that for any object $z$,
    \begin{gather*}
        z \in\set{y : P(x, y)\text{ is true for some }x\in A}\\
        \iff P(x,z)\text{ is true for some }x\in A
    \end{gather*}

    \textbf{Axiom 3.9} (Universal specification). (Dangerous!) Suppose for every object x we
    have a property $P(x)$ pertaining to $x$ (so that for every $x$, $P(x)$ is either a true
    statement or a false statement). Then there exists a set $\set{x : P(x)\ \text{is true}}$ such that
    for every object $y$,
    \[y \in\set{x : P(x)\ \text{is true}}\iff P(y)\ \text{is true}\]
\end{quotation}
\begin{proof}
    Axiom 3.3 (Empty set):
    Let $P(x)$ be the property that $x\neq x$.
    \[
        y\in\left\{x:x\neq x\ \textrm{is true}\right\}\iff y\neq y\ \textrm{is true}.
    \]

    Axiom 3.4 (Singleton sets \st{and pair sets}):
    Let $P(x)$ be the property that $x=a$.
    \[
        y\in\left\{x:x=a\ \textrm{is true}\right\}\iff y=a\ \textrm{is true}.
    \]

    Axiom 3.5 (Pairwise union):
    Let $P(x)$ be the property that $x\in A$ or $x\in B$.
    \[
        y\in\left\{x:x\in A\ \textrm{or}\ x\in B\ \textrm{is true}\right\}\iff y\in A\ \textrm{or}\ y\in B\ \textrm{is true}.
    \]

    Axiom 3.6 (Axiom of specification). Let $P(x)$ be the property that $x\in A$ and $P(x)$ is true.
    \[
        y\in\left\{x:x\in A\ \textrm{and}\ P(x)\ \textrm{is true}\right\}\iff y\in A\ \textrm{and}\ P(y)\ \textrm{is true}.
    \]

    Axiom 3.7 (Replacement) Let $P(y)$ be the property that there exists $x\in A$ such that $P(x,y)$ is true.
    \[
        z\in\left\{y:P(y)\ \textrm{is true}\right\}=\left\{y:P(x,y)\ \textrm{is true}\ \textrm{for some}\ x\in A\right\}\\
        \iff P(x,z)\ \textrm{is true}\ \textrm{for some}\ x.
    \]
\end{proof}

\section{Exercise 3.2.2}
Use the axiom of regularity (and the singleton set axiom) to show that if $A$ is a set, then $A\notin A$. Furthermore, show that if $A$ and $B$ are two sets, then either $A\notin B$ or $B\notin A$ (or both).
(One corollary of this exercise is worth noting: given any set $A$, there exists a mathematical object
that is not an element in $A$, namely $A$ itself. Thus one can always “add one more element” to a set
$A$ to create a larger set, namely $A \cup\set{A}$.)
\begin{quotation}

    \textbf{Axiom 3.10} (Regularity). If $A$ is a non-empty set, then there is at least one element
    $x$ of $A$ which is either not a set, or is disjoint from $A$.
\end{quotation}
\begin{proof}
    Suppose that $A$ is a non-empty set, and $A$ is an element of $A$.

    Let's consider set $\set{A}$, which can be constructed through Axiom 3.4:

    Obviously, $A\in\set{A}$

    By Axiom 3.10, $A$ is either not a set, or is disjoint from $\set{A}$, which leads to $A\cap\set{A}=\emptyset$.

    Since $A\in A$ and $A\in\set{A}$, this implies that $A\cap\set{A}=A$, a contradiction. While if $A$ is an empty set. By Axiom 3.3, $A\notin A$.

    Furthermore, suppose that $A,B$ are two sets and we both have $A\in B$ and $B\in A$. By Axiom 3.4 (and Axiom 3.5), we have $A,B\in\set{A,B}$. By Axiom 3.10, we have $A\cap\set{A,B}=\emptyset$ and $B\cap\set{A,B}=\emptyset$, but this is contradictive with $A\in B,B\in A$ and $A,B\in\set{A,B}$.
\end{proof}

\section{Exercise 3.2.3}
Show (assuming the other axioms of set theory) that the universal specification
axiom, Axiom 3.9, is equivalent to an axiom postulating the existence of a “universal set” $\Omega$
consisting of all objects (i.e., for all objects $x$,we have $x \in \Omega$). In other words, if Axiom 3.9 is true,
then a universal set exists, and conversely, if a universal set exists, then Axiom 3.9 is true. (This
helps explain why Axiom 3.9 is called the axiom of universal specification.) Note that if a universal
set
existed, then we would have $\Omega\in\Omega$ by Axiom 3.1, contradicting Exercise 3.2.2. Thus the
axiom of foundation specifically rules out the axiom of universal specification.
\begin{proof}
    Suppose that Axiom 3.8 it true, then we have
    $y\in\{x:x\in\Omega\}$.
    Conversely, if a universal set $\Omega$ exists, by Axiom 3.6, we have
    $y\in\{x\in\Omega:P(x)\ \textrm{is true}\}\implies y\in\{x:P(x)\ \textrm{is true}\}$.
    Hence we obtain Axiom 3.8.
\end{proof}

\section{Thought on the set of all sets}
Let $U$ be the set of all sets. Does such $U$ exist?
(Note that $U$ is different from the universal set $\Omega$ in the Exercise 3.2.3,
where $\Omega$ is the set consisting everything.)
\begin{proof}
    The proof is clipped from \href{https://pansci.asia/archives/75290}{PanSci}.
    We slightly modify the example in Russell’s Paradox ($\set{x|x\notin x}$) to
    \[B:=\set{x\in U|x\notin x}\]
    If $B\in B$, the by the definition of $B$, there is
    \[B\in B\implies B\in V \text{ and }B\notin B\tag{1}\]
    If $B\notin B$, since $U$ contains every set, we have $B\in V$,
    so $B$ complies the definition of B, which means
    \[B\in U\text{ and }B\notin B\implies B\in B\tag{2}\]
    With (1) and (2), we get
    \[B\in B\iff B\in U\text{ and }B\notin B\]
    This made the claim above to be true only when it's vacuously true,
    which is that $B\in V$ is not true. Thus lead to contradiction.
\end{proof}

\section{Exercise 3.4.2 (iii)}
Solution found from \url{https://math.stackexchange.com/a/4710434}

\textbf{Exercise 3.4.2} Let $f : X \to Y$ be a function from one set $X$ to another set $Y$,let $S$ be a subset of
$X$,and let $U$ be a subset of $Y$.

(i) What, in general, can one say about $f^{-1}( f (S))$ and $S$?

(ii) What about $f( f^{-1}(U))$ and $U$?

(iii) What about $f^{-1}( f ( f^{-1}(U)))$ and $f^{-1}(U)$?

\begin{proof}
    For (i) and (ii), there's solution in answer book.

    For (iii), the anwser is $f^{-1}( f ( f^{-1}(U)))=f^{-1}(U)$

    From (i), we get $f^{-1}(U)\subseteq f^{-1}( f ( f^{-1}(U)))$. So we only need to prove
    $f^{-1}( f ( f^{-1}(U)))\subseteq f^{-1}(U)$

    For any $x\in f^{-1}( f ( f^{-1}(U)))$, by definition,
    we have $f(x)\in f ( f^{-1}(U))$,
    so there exists $z\in f^{-1}(U)$ such that $f(x)=f(z)$.
    Now, since $z\in f^{-1}(U)$ then $f(z)\in U$,
    hence $f(x)\in U$. Therefore, $x\in f^{-1}(U)$.
\end{proof}

\section{Exercise 3.4.6 (ii)}
\textbf{Exercise 3.4.6} (ii)
Conversely, show that Axiom 3.11 can be deduced the preceding axioms of set theory if one
accepts Lemma 3.4.10 as an axiom. (This may help explain why we refer to Axiom 3.11 as
the “power set axiom”.)

\begin{quotation}
    \textbf{Axiom 3.11} (Power set axiom).
    Let $X$ and $Y$ be sets.Then there exists a set,denoted
    $Y^X$, which consists of all the functions from $X$ to $Y$, thus
    \[f \in Y^X \iff (f \text{ is a function with domain }X\text{ and codomain }Y)\]

    \textbf{Lemma 3.4.10} Let $X$ be a set. Then the set
    \[\set{Y : Y \text{ is a subset of }X}\]
    is a set. That is to say, there exists a set $Z$ such that
    $Y \in Z\iff Y \subseteq X$
    for all objects $Y$
\end{quotation}
\begin{proof}
    For (i) there's solution in answer book.

    From \url{https://math.stackexchange.com/a/409659}

    Since $X$ and $Y$ exist, thus $A\times B$ exists, so $\mathcal{P}(A\times B)$
    exists (see exercise 3.5.1). Thus by axiom 3.6 (Axiom of specification), we have that
    \[\set{f\in \mathcal{P}(X\times Y):f\text{ is a function }X\to Y}\]
    exists. But this is just $Y^X$, by definition. So we're done.
\end{proof}

\section{Exercise 3.5.1}
 (i) Suppose we \textit{define} the ordered pair $(x, y)$
for any objects $x$ and $y$ by the formula $(x, y) := \{\{x\}, \{x, y\}\}$
(thus using several applications of Axiom 3.4).
Thus, for instance, $(1,2)$ is the set $\{\{1\}, \{1,2\}\}$,
$(2,1)$ is the set $\{\{2\}, \{2,1\}\}$,
and $(1,1)$ is the set $\{\{1\}\}$.
Show that such a definition (known as the \textit{Kuratowski definition} of an ordered pair)
indeed obeys the property (3.5).

(ii) Suppose we instead define an ordered pair using the alternate definition $(x, y) := \{x, \{x, y\}\}$.
Show that this definition (known as the \textit{short definition} of an ordered pair)
also verifies (3.5) and is thus also an acceptable definition of ordered pair.
(Warning: this is tricky; one needs the axiom of regularity, and in particular Exercise 3.2.2.)

(iii) Show that regardless of the definition of ordered pair,
the Cartesian product $X \times Y$ of any two sets $X, Y$ is again a set.
(Hint: first use the axiom of replacement to show that for any $x \in X$,
the set $\{(x, y) : y \in Y\}$ is a set, and then apply the axiom of union.)

\begin{quotation}
    Axiom 3.4 is quoted in section Exercise 3.2.1.
    And Axiom of regularity is quoted in section Exercise 3.2.2.
    They are not quoted here again.

    \textbf{Axiom 3.2} (Equality of sets).
    Two sets $A$ and $B$ are equal, $A = B$,
    iff every element of $A$ is an element of $B$ and vice versa.
    To put it another way, $A = B$ if and only if every element $x$ of $A$ belongs also to $B$,
    and every element $y$ of $B$ belongs also to $A$.

    \textbf{Axiom 3.12} (Union). Let $A$ be a set, all of whose elements are themselves sets.
    Then there exists a set $\bigcup A$
    whose elements are precisely those objects which are elements of the elements of $A$,
    thus for all objects $x$,
    \[
        x \in\bigcup A \iff (x \in S \text{ for some } S \in A).
    \]

    \textbf{Definition 3.5.1} (Ordered pair). If $x$ and $y$ are any objects (possibly equal),
    we define the \textit{ordered pair} $(x, y)$ to be a new object,
    consisting of $x$ as its first component and $y$ as its second component.
    Two ordered pairs $(x, y)$ and $(x', y')$ are considered equal
    if and only if both their components match, i.e.,
    \[
        (x, y) = (x', y') \iff (x = x' \text{ and } y = y')\tag*{(3.5)}
    \]
\end{quotation}

There is solution in other anwser book, but it doesn't use the hint.
The proof below utilizes the hint.

\begin{proof}
    (i) If $(x, y) = (x', y')$, then by Kuratowski definition we have
    \[\set{x}=\set{x'}\text{ and }\set{x,y}=\set{x',y'}\]
    or
    \[\set{x}=\set{x',y'}\text{ and }\set{x,y}=\set{x'}\]
    In the first case, we have $x=x'$ and whatever $\set{x,y}$, $\set{x',y'}$
    are singleton or pairwise set, we all have $y=y'$.
    For the second case, $\set{x,y}$ and $\set{x',y'}$ have to be singleton set,
    then we can easily see that $x=x'=y=y'$.

    Conversely, if $x=x'$ and $y=y'$, then by axiom 3.2, we can easily conclude $(x, y) = (x', y')$.

    (ii) if $(x, y) = (x', y')$, then by short definition we have
    \[\set{x,\set{x,y}}=\set{x',\set{x',y'}}\]
    We first show that $\set{x,\set{x,y}}$ or $\set{x',\set{x',y'}}$ can't be singleton set.
    If $\set{x,\set{x,y}}$ is a singleton set, then we have $x=\set{x,y}$.
    So $x$ is a set, by axiom of regularity, we have $x$ or $\set{x,y}$ is disjoint from $\set{x,\set{x,y}}$.
    If $x=\set{x,y}$, we have $\set{x,y}=\set{\set{x,y},y}$,
    but $\set{x,y}\in\set{\set{x,y},y}\cap\set{x,\set{x,y}}$ will lead to contradiction.

    Now there remains two cases:
    \begin{itemize}
        \item Case 1: $x=x'$ and $\set{x,y}=\set{x',y'}$.
        \item Case 2: $\set{x,y}=x'$ and $x=\set{x',y'}$.
    \end{itemize}
    If it's the second case, we have $x$ and $\set{x,y}$ are sets.
    Let's examine if $\set{x,{x,y}}$ complies axiom of regularity.

    Since $x=\set{x',y'}=\set{\set{x,y},y'}$, $\set{x,y}\in x\cap\set{x,\set{x,y}}$.
    Besides, $x\in\set{x,y}\cap\set{x,\set{x,y}}$.
    So for the second case, there's a contradiction.
    For the first case, it's easy to verify $x=x'$ and $y=y'$.

    (iii) We follow the hint. Fix $x$, for any $y\in Y$, we can construct a property $P(y,z)$ such that
    \begin{itemize}
        \item Definition 3.5.1: Let $P(y,z)$ be $z=(x,y)$.
        \item Kuratowski definition: Let $P(y,z)$ be $z=\{\{x\},\{x,y\}\}$.
        \item Short definition: Let $P(y,z)$ be $z=\{x,\{x,y\}\}$.
    \end{itemize}
    In short, for a fixed $x$, we have constructed $\set{(x,y):y\in Y}$ as a set.
    For every $x\in X$, we can construct such set.
    Then we union all of these sets together, we have $X\times Y$, which by Axiom of union, is a set
\end{proof}

\section{Note on exercise 3.5.6}
\begin{quotation}
    \textbf{Exercise 3.5.6}
    Let $A, B, C, D$ be non-empty sets. Show that $A \times B \subseteq C \times D$ if and only if $A \subseteq C$ and $B \subseteq D$, and that $A \times B = C \times D$ if and only if $A = C$ and $B = D$.
    What happens if some or all of the hypotheses that $A, B, C, D$ are non-empty are removed?
\end{quotation}
Note that the hypotheses of non-emptiness cannot be removed. Think about the case
$A=\emptyset $ and $B\ne \emptyset $, and $A\times B\subseteq C\times D$,
then $A\subseteq C$ and $B$ can be any set, and $D$ can be any set.

\section{Exercise 3.5.9}
Suppose that $I$ and $J$ are two sets, and for all $\alpha \in I$, let $A_\alpha$ be a set,
and for all $\beta \in J$, let $B_\beta$ be a set. Show that
$\left( \bigcup_{\alpha \in I} A_\alpha \right)
    \cap \left( \bigcup_{\beta \in J} B_\beta \right)
    = \bigcup_{(\alpha, \beta) \in I \times J} (A_\alpha \cap B_\beta)$
What happens if one interchanges all the union and intersection symbols here?

\begin{proof}
    The First part of the exercise is shown in the answer book.
    For the case after interchanging all the union and intersection symbols,
    the equation still holds.

    What we want to prove is that
    \[
        (\bigcup_{\alpha \in I} A_{\alpha})
        \cap (\bigcup_{\beta \in J} B_{\beta})
        = \bigcup_{(\alpha,\beta) \in I \times J} (A_{\alpha} \cap B_{\beta})
    \]
    For the "$\implies$" direction,
    if $x\in (\bigcup_{\alpha \in I} A_{\alpha})\cap (\bigcup_{\beta \in J} B_{\beta})$,
    then there exists $\alpha'\in I$ such that $x\in A_{\alpha'}$,
    and $\beta'\in J$ such that $x\in B_{\beta'}$.
    So we have $x\in A_{\alpha'}\cap B_{\beta'}$, which implies
    there's $(\alpha',\beta')\in I\times J$ such that $x\in A_{\alpha'}\cap B_{\beta'}$.
    Thus the "$\implies$" direction is proved.

    For the "$\impliedby$" direction, if $x\in \bigcup_{(\alpha,\beta) \in I \times J} (A_{\alpha} \cap B_{\beta})$,
    then there exists $(\alpha',\beta')\in I\times J$ such that $x\in A_{\alpha'}\cap B_{\beta'}$.
    So $x\in A_{\alpha'}$ and $x\in B_{\beta'}$,
    which implies $x\in\bigcup_{\alpha \in I} A_{\alpha}$ and $x\in\bigcup_{\beta \in J} B_{\beta}$.
    Thus $x\in (\bigcup_{\alpha \in I} A_{\alpha})\cap (\bigcup_{\beta \in J} B_{\beta})$.
    The "$\impliedby$" direction is proved.
\end{proof}

\section{Note on exercise 3.5.12}

\begin{quotation}
    \textbf{Proposition 2.1.16} (Recursive definitions).
    Suppose for each natural number $n$, we have some function $f_n : \mathbb{N} \to \mathbb{N}$
    from the natural numbers to the natural numbers.
    Let $c$ be a natural number.
    Then we can assign a unique natural number $a_n$ to each natural number $n$, such that
    $a_0 = c$ and $a_{n\doubleplus} = f_n(a_n) \quad \text{for each natural number } n$.
    \begin{proof}
        (Informal) We use induction.
        We first observe that this procedure gives a single value to $a_0$, namely $c$.
        (None of the other definitions $a_n\doubleplus := f_n(a_n)$ will redefine the value of $a_0$,
        because of Axiom 2.3.)
        Now suppose inductively that the procedure gives a single value to $a_n$.
        Then it gives a single value to $a_n\doubleplus$, namely $a_n\doubleplus := f_n(a_n)$.
        (None of the other definitions $a_m\doubleplus := f_m(a_m)$ will redefine the value of $a_n\doubleplus$,
        because of Axiom 2.4.)
        This completes the induction, and so $a_n$ is defined for each natural number $n$,
        with a single value assigned to each $a_n$.
    \end{proof}

    \textbf{Exercise 3.5.12}

    This exercise will establish a rigorous version of Proposition 2.1.16 that avoids circularity
    (in particular, avoiding the use of any object that required Proposition 2.1.16 to construct).

    (i) Let $X$ be a set, let $f : \mathbb{N} \times X \to X$ be a function,
    and let $c$ be an element of $X$. Show that there exists a function $a: \mathbb{N} \to X$ such that
    \[
        a(0) = c
    \]
    and
    \[
        a(n+1) = f(n, a(n)) \quad \text{for all } n \in \mathbb{N}
    \]
    and furthermore that this function is unique.

    (Hint: first show inductively, by a modification of the proof of Lemma 3.5.11,
    that for every natural number $N \in \mathbb{N}$,
    there exists a unique function $a_N : \{n \in \mathbb{N} : n \leq N\} \to X$
    such that $a_N(0) = c$ and $a_N(n+1) = f(n, a_N(n))$ for all $n \in \mathbb{N}$ such that $n < N$.)

    (ii) (Warning: this is challenging.)
    Prove (i) without using any properties of the natural numbers other than the Peano axioms directly
    (in particular, without using the ordering of the natural numbers,
    and without appealing to Proposition 2.1.16).

    (Hint: first show inductively, using only the Peano axioms and basic set theory,
    that for every natural number $N \in \mathbb{N}$,
    there exists a unique pair $A_N, B_N$ of subsets of $\mathbb{N}$ which obeys the following properties:
    (a) $A_N \cap B_N = \emptyset$, (b) $A_N \cup B_N = \mathbb{N}$, (c) $0 \in A_N$,
    (d) $N\doubleplus \in B_N$, (e) whenever $n \in B_N$, we have $n\doubleplus \in B_N$,
    (f) whenever $n \in A_N$ and $n \neq N$, we have $n\doubleplus \in A_N$.
    Once one obtains these sets, use $A_N$ as a substitute for
    $\{n \in \mathbb{N} : n \leq N\}$ in the previous argument.)
\end{quotation}

Why do the author say that the version of this exercise is more
rigorous than the proof in proposition 2.1.16?
In \href{https://terrytao.wordpress.com/books/analysis-i/comment-page-15/}{Tao's word}, it's because

\begin{quotation}
    The informal argument given for Proposition 2.1.16 actually cannot be
    written formally in the language of first-order logic without first
    using something like a partial function to formalize what it means to “uniquely define” a quantity $a_n$.
    The problem here is that the uniqueness statement involves not just the value of $a_n$,
    but also all preceding values $a_0, a_1, \dots, a_{n-1}$.
    To be able to refer to these values formally, one needs a partial function.
\end{quotation}

As we are not researching logic, we won't take too much care about this.

For full proof, please refer to \url{https://taoanalysis.wordpress.com/2020/05/04/exercise-3-5-12/}

Note that there is a typo mistake in (i) in the book. $a$'s domain should be $\mathbb{N}$.
Quatation below contains the correct version.

\section{Exercise 3.5.13}
The purpose of this exercise is to show that there is essentially only one version
of the natural number system in set theory (cf. the discussion in Remark 2.1.12).

Suppose we have a set $\mathbb{N}'$ of “alternative natural numbers”, an “alternative zero”
$0'$, and an “alternative increment operation” which takes any alternative natural
number $n' \in \mathbb{N}'$ and returns another alternative natural number $n'\doubleplus' \in \mathbb{N}'$, such
that the Peano axioms (Axioms 2.1–2.5) all hold with the natural numbers, zero,
and increment replaced by their alternative counterparts.
Show that there exists a bijection $f : \mathbb{N} \to \mathbb{N}'$ from the natural numbers
to the alternative natural numbers such that $f(0) = 0$, and such that for any
$n \in \mathbb{N}$ and $n' \in \mathbb{N}'$, we have
$f(n) = n$ if and only if $f(n\doubleplus) = n'\doubleplus'$.
(Hint: use Exercise 3.5.12.)

The proof is revised from the one in
\href{https://christangdt.home.blog/2019/04/04/%e9%99%b6%e5%93%b2%e8%bd%a9%e5%ae%9e%e5%88%86%e6%9e%903-5%e5%8f%8a%e4%b9%a0%e9%a2%98-analysis-i-3-5/}{{christangdt's blog}}
to make use of exercise 3.5.12.
\begin{proof}
    Let $g:\mathbb{N}\times\mathbb{N}'\to\mathbb{N}'$ be a function that satisfies
    \[g(n,n')=n'\doubleplus'\]
    and $f:\mathbb{N}\to\mathbb{N}'$ be a function such that
    \[f(0) = 0', \quad f(n\doubleplus) = g(n,f(n)), \quad \forall n \in \mathbb{N}\]
    By exercise 3.5.12, $f$ is a unique function satisfies the above condition.
    Besides, we can conclude $f(n\doubleplus)=f(n)\doubleplus'$ and
    $f(n) = n' \iff f(n\doubleplus) = n'\doubleplus'$.

    To show $f$ is injective: Let $f(n) = f(m)$ for $m, n \in \mathbf{N}$. If $f(n) = f(m) = 0'$, then we have $m = n = 0$.
    If $f(n) = f(m) = n' \neq 0'$, then obviously $n \neq 0, m \neq 0$, and we have
    \[
        a, b \in \mathbf{N}, \quad a\doubleplus = n, \quad b\doubleplus = m, \quad f(a)\doubleplus' = f(b)\doubleplus'.
    \]
    This means $f(a) = f(b)$. Now assume $n \neq m$. Without loss of generality, let $n = 0$. Then we can deduce that
    \[
        f(0) = f(d) = 0'.
    \]
    As $d > 0$, there exists $e \in \mathbf{N}$ such that $e\doubleplus = d$, then
    \[
        f(d) = f(e)\doubleplus' = 0'.
    \]
    This violates Peano Axiom 3.

    To show $f$ is surjective: Given $\forall n' \in \mathbb{N}'$, if $n' = 0'$, we have $f(0) = n'$. If $n' > 0'$, there exists $m' \in \mathbb{N}'$ such that $m'\doubleplus' = n'$. Using induction, we assume that for every alternative natural number less than $n'$, there exists a natural number whose image is the given alternative natural number, so there exists $m \in \mathbf{N}$ such that $f(m) = m'$. Thus,
    \[
        f(m\doubleplus) = f(m)\doubleplus' = n',
    \]
    and $m\doubleplus \in \mathbf{N}$, which completes the proof.
\end{proof}

\section{Exercise 7.1.7}

\textbf{Exercise 7.1.7}
Let \( n, m \) be natural numbers, and for each \( 1 \leq i \leq n \), let \( a_i \) be a natural number with \( a_i \leq m \). Establish the identity:
\[
    \sum_{i=1}^{n} a_i = \sum_{j=1}^{m} \#\left(\{1 \leq i \leq n : a_i \geq j\}\right).
\]
\textit{(Hint: apply Corollary 7.1.14 to compute a sum \( \sum_{i=1}^{n} \sum_{j=1}^{m} c_{i,j} \) in two different ways, for a well-chosen choice of summands \( c_{i,j} \).)}

Use of identities such as this is known as the double counting method, and is often useful in combinatorics.


\begin{proof}
    The proof is generated by Genimi.

    The goal is to prove the identity:
    \(
    \sum_{i=1}^{n} a_i = \sum_{j=1}^{m} \#\{1 \leq i \leq n : a_i \geq j\}
    \)

    This method involves defining a quantity and then summing it in two different orders to arrive at the left-hand side (LHS) and right-hand side (RHS) of the identity.

    \textbf{Step 1:} Define the Summand

    As the hint suggests, we need to choose a summand \( c_{i,j} \) to use in a double summation. A strategic choice is to define \( c_{i,j} \) based on the condition in the right-hand side of the identity (\( a_i \geq j \)).

    Let's define \( c_{i,j} \) for each pair of indices \( (i, j) \) where \( 1 \leq i \leq n \) and \( 1 \leq j \leq m \) as follows:
    \[
        c_{i,j} = \begin{cases}
            1 & \text{if } j \leq a_i \\
            0 & \text{if } j > a_i
        \end{cases}
    \]

    Now, consider the total sum \( S = \sum_{i=1}^{n} \sum_{j=1}^{m} c_{i,j} \). We will compute this sum in two ways by changing the order of summation.

    \textbf{Step 2:} Calculate the Sum (First Way)

    First, we'll sum over \( j \) and then over \( i \). This corresponds to the left-hand side of the identity.

    We have
    \(
    S = \sum_{i=1}^{n} \left( \sum_{j=1}^{m} c_{i,j} \right)
    \)

    Let's evaluate the inner sum for a fixed \( i \):
    \(
    \sum_{j=1}^{m} c_{i,j}
    \)

    According to our definition, \( c_{i,j} \) is 1 only when \( j \leq a_i \). Since the problem states \( a_i \leq m \), the sum becomes
    \(
    \sum_{j=1}^{a_i} 1 + \sum_{j=a_i+1}^{m} 0 = a_i + 0 = a_i
    \)

    Now, we substitute this back into the outer sum:
    \(
    S = \sum_{i=1}^{n} a_i
    \)

    This is exactly the \textbf{left-hand side} of the identity.

    \textbf{Step 3:} Calculate the Sum (Second Way)

    Next, we'll switch the order of summation: sum over \( i \) first, and then over \( j \). This will correspond to the right-hand side.

    We have
    \(
    S = \sum_{j=1}^{m} \left( \sum_{i=1}^{n} c_{i,j} \right)
    \)

    Let's evaluate the inner sum for a fixed \( j \):
    \(
    \sum_{i=1}^{n} c_{i,j}
    \)

    According to our definition, \( c_{i,j} \) is 1 only when \( a_i \geq j \). So, when we sum over all possible values of \( i \) (from 1 to \( n \)), we are simply counting how many of the numbers \( a_1, a_2, \dots, a_n \) are greater than or equal to \( j \).

    This count is precisely what the notation \( \#\{1 \leq i \leq n : a_i \geq j\} \) represents. Therefore:
    \(
    \sum_{i=1}^{n} c_{i,j} = \#\{1 \leq i \leq n : a_i \geq j\}
    \)

    Now, we substitute this back into the outer sum:
    \(
    S = \sum_{j=1}^{m} \#\{1 \leq i \leq n : a_i \geq j\}
    \)

    This is exactly the \textbf{right-hand side} of the identity.
\end{proof}

\textit{Remark:} We can imagine in this way, there are several cubes stacked in several
piles, and the height of the $i$-th pile is $a_i$. We can either count the total number of cubes
pile by pile or layer by layer.

\includegraphics[width=0.45\textwidth]{count-by-layer.png}
\hfil
\includegraphics[width=0.45\textwidth]{count-by-pile.png}


\section{Limit comparison test}
Corollary 7.3.2 stated the direct comparison test for series.
\begin{quotation}
    \textbf{Corollary 7.3.2} (Comparison test) Let $\sum_{n=m}^{\infty} a_n$ and
    $\sum_{n=m}^{\infty} b_n$ be two formal series of real numbers, and suppose
    that $|a_n| \leq b_n$ for all $n \geq m$.
    Then if $\sum_{n=m}^{\infty} b_n$ is convergent, then $\sum_{n=m}^{\infty} a_n$
    is absolutely convergent, and in fact
    \[
        \sum_{n=m}^{\infty} a_n \leq \sum_{n=m}^{\infty} |a_n| \leq \sum_{n=m}^{\infty} b_n.
    \]
\end{quotation}
There is another useful test for series, known as the limit comparison test, but not mentioned in Tao's book.
\begin{quotation}
    \textbf{Limit comparison test}
    Let $a_n \geq 0$, $b_n > 0$ for all $n$. Then if
    $\lim_{n \to \infty} \frac{a_n}{b_n} = c$ with $0 < c < \infty$,
    either both series $\sum_{n=m}^{\infty} a_n$ and
    $\sum_{n=m}^{\infty} b_n$ converge, or both diverge.
\end{quotation}
\begin{proof}
    Because $\lim_{n \to \infty} \frac{a_n}{b_n} = c$, we know that for every
    $\varepsilon > 0$, there exists a positive integer $n_0$ such that for all
    $n \geq n_0$ we have
    $\left| \frac{a_n}{b_n} - c \right| < \varepsilon$, or equivalently,
    $-\varepsilon < \frac{a_n}{b_n} - c < \varepsilon$, hence
    $c - \varepsilon < \frac{a_n}{b_n} < c + \varepsilon$.

    Multiplying through by $b_n$, we get
    \[
        (c - \varepsilon)b_n < a_n < (c + \varepsilon)b_n.
    \]

    Since $c > 0$, we can choose $\varepsilon$ small enough so that
    $c - \varepsilon > 0$. Then we also have
    \[
        b_n < \frac{1}{c - \varepsilon} a_n.
    \]

    By the Direct Comparison Test, if $\sum a_n$ converges, then so does
    $\sum b_n$.

    Similarly, from $a_n < (c + \varepsilon)b_n$, if $\sum a_n$ diverges,
    then so does $\sum b_n$.

    Thus, both series converge or both diverge.
\end{proof}

\section{Exercise 7.4.2}
Obtain an alternate proof of Proposition 7.4.3 using Proposition 7.4.1,
Proposition 7.2.13, and expressing $a_n$ as the difference of $(a_n + |a_n|)$
and $|a_n|$. (This argument is due to Will Ballard.)

\begin{quotation}
    \textbf{Proposition 7.4.3} (Rearrangement of series) Let $\sum_{n=0}^{\infty} a_n$
    be an absolutely convergent series of real numbers, and let $f: \mathbb{N} \to \mathbb{N}$
    be a bijection.
    Then $\sum_{m=0}^{\infty} a_{f(m)}$ is also absolutely convergent, and has the
    same sum:
    \[
        \sum_{n=0}^{\infty} a_n = \sum_{m=0}^{\infty} a_{f(m)}.
    \]

    \textbf{Proposition 7.4.1} Let $\sum_{n=0}^{\infty} a_n$ be a convergent series
    of non-negative real numbers, and let $f: \mathbb{N} \to \mathbb{N}$ be a bijection.
    Then $\sum_{m=0}^{\infty} a_{f(m)}$ is also convergent, and has the same sum:
    \[
        \sum_{n=0}^{\infty} a_n = \sum_{m=0}^{\infty} a_{f(m)}.
    \]

    \textbf{Proposition 7.2.13} (Series laws)

    (a) If $\sum_{n=m}^{\infty} a_n$ is a series of real numbers converging to $x$,
    and $\sum_{n=m}^{\infty} b_n$ is a series of real numbers converging to $y$, then
    $\sum_{n=m}^{\infty} (a_n + b_n)$ is also a convergent series, and converges to $x + y$.
    In particular, we have
    \[
        \sum_{n=m}^{\infty} (a_n + b_n) = \sum_{n=m}^{\infty} a_n + \sum_{n=m}^{\infty} b_n.
    \]
\end{quotation}

\begin{proof}
    Let $\sum_{n=0}^\infty a_n$ be absolutely convergent, so in particular $\sum |a_n|$ converges. \
    Define two new series by
    \[
        b_n = a_n + |a_n|, \quad c_n = |a_n|.
    \]
    Then for each \( n \), \( b_n \geq 0 \) and \( c_n \geq 0 \).
    Moreover,
    \[
        \sum_{n=0}^\infty b_n = \sum_{n=0}^\infty (a_n + |a_n|) = \sum_{n=0}^\infty a_n + \sum_{n=0}^\infty |a_n|
    \]
    converges (as the sum of two convergent series), and clearly \( \sum c_n = \sum |a_n| \) converges.

    Let \( f: \mathbb{N} \to \mathbb{N} \) be any bijection.
    Since both \( \{b_n\} \) and \( \{c_n\} \) are nonnegative-term series,
    proposition 7.4.1 applies to each, giving
    \[
        \sum_{n=0}^\infty b_n = \sum_{m=0}^\infty b_{f(m)}, \quad \sum_{n=0}^\infty c_n = \sum_{m=0}^\infty c_{f(m)}.
    \]
    But \( b_{f(m)} = a_{f(m)} + |a_{f(m)}| \) and \( c_{f(m)} = |a_{f(m)}| \).
    Hence
    \[
        \sum_{m=0}^\infty b_{f(m)} - \sum_{m=0}^\infty c_{f(m)} = \sum_{m=0}^\infty \bigl(a_{f(m)} + |a_{f(m)}|\bigr) - \sum_{m=0}^\infty |a_{f(m)}|.
    \]
    Since both of these latter series converge,
    Proposition 7.2.13 (on the difference of two convergent series)
    tells us that their difference converges and equals the difference of their sums.
    Therefore,
    \[
        \sum_{m=0}^\infty \bigl(a_{f(m)} + |a_{f(m)}|\bigr) - \sum_{m=0}^\infty |a_{f(m)}| = \sum_{m=0}^\infty a_{f(m)}.
    \]
    Putting everything together,
    \[
        \sum_{n=0}^\infty a_n + \sum_{n=0}^\infty |a_n| = \sum_{n=0}^\infty b_n = \sum_{m=0}^\infty b_{f(m)} = \sum_{m=0}^\infty a_{f(m)} + \sum_{m=0}^\infty |a_{f(m)}|
    \]
    and
    \[
        \sum_{n=0}^\infty |a_n| = \sum_{m=0}^\infty |a_{f(m)}|.
    \]
    Subtracting the second equality from the first (again by \textbf{Proposition 7.2.13}) yields
    \[
        \sum_{n=0}^\infty a_n = \sum_{m=0}^\infty a_{f(m)},
    \]
    as required.
\end{proof}




\section{Takeaway}
\textbf{Corollary 6.5.1} We have
\(\lim_{n \to \infty} \frac{1}{n^{1/k}} = 0\) for every integer \( k \geq 1\).

\medskip

\textbf{Lemma 6.5.2} Let $x$ be a real number. Then the limit
$\lim_{n \to \infty} x^n$ exists and is equal to $0$ when $|x| < 1$,
exists and is equal to $1$ when $x = 1$, and diverges when $x = -1$
or when $|x| > 1$.

\medskip

\textbf{Lemma 6.5.3} For any $x > 0$, we have
\(
\lim_{n \to \infty} x^{1/n} = 1.
\)

\medskip

\textbf{Proposition 7.2.5} Let $\sum_{n=m}^{\infty} a_n$ be a formal series of real numbers.
Then $\sum_{n=m}^{\infty} a_n$ converges if and only if, for every real number
$\varepsilon > 0$, there exists an integer $N \geq m$ such that

\[
    \sum_{n=p}^{q} a_n \leq \varepsilon \quad \text{for all } p, q \geq N.
\]

\medskip

\textbf{Corollary 7.2.6} (Zero test) Let $\sum_{n=m}^{\infty} a_n$ be a
convergent series of real numbers.
Then we must have
\(
\lim_{n \to \infty} a_n = 0.
\)
To put this another way, if $\lim_{n \to \infty} a_n$ is non-zero or
divergent, then the series $\sum_{n=m}^{\infty} a_n$ is divergent.

\medskip

\textbf{Definition 7.2.8} (Absolute convergence) Let $\sum_{n=m}^{\infty} a_n$
be a formal series of real numbers.
We say that $\sum_{n=m}^{\infty} a_n$ converges absolutely if the series
$\sum_{n=m}^{\infty} |a_n|$ converges.

\medskip

\textbf{Proposition 7.2.9} (Absolute convergence test) Let $\sum_{n=m}^{\infty} a_n$
be a formal series of real numbers.
If this series is absolutely convergent, then it is also convergent. Furthermore,
in this case we have the triangle inequality:
\[
    \sum_{n=m}^{\infty} a_n \leq \sum_{n=m}^{\infty} |a_n|.
\]

\medskip

\textbf{Proposition 7.2.11} (Alternating series test) Let $(a_n)_{n=m}^{\infty}$
be a sequence of real numbers which are non-negative and decreasing,
thus $a_n \geq 0$ and $a_n \geq a_{n+1}$ for every $n \geq m$.
Then the series $\sum_{n=m}^{\infty} (-1)^n a_n$ is convergent if and
only if the sequence $a_n$ converges to $0$ as $n \to \infty$.

\medskip

\textbf{Proposition 7.3.1} Let $\sum_{n=m}^{\infty} a_n$ be a formal series of
non-negative real numbers.
Then this series is convergent if and only if there is a real number $M$ such that
\[
    \sum_{n=m}^{N} a_n \leq M \quad \text{for all integers } N \geq m.
\]

\medskip

\textbf{Corollary 7.3.2} (Comparison test) Let $\sum_{n=m}^{\infty} a_n$ and
$\sum_{n=m}^{\infty} b_n$ be two formal series of real numbers, and suppose
that $|a_n| \leq b_n$ for all $n \geq m$.
Then if $\sum_{n=m}^{\infty} b_n$ is convergent, then $\sum_{n=m}^{\infty} a_n$
is absolutely convergent, and in fact
\[
    \sum_{n=m}^{\infty} a_n \leq \sum_{n=m}^{\infty} |a_n| \leq \sum_{n=m}^{\infty} b_n.
\]

\medskip

\textbf{Lemma 7.3.3} (Geometric series) Let $x$ be a real number.
If $|x| \geq 1$, then the series $\sum_{n=0}^{\infty} x^n$ is divergent.
However, if $|x| < 1$, then the series is absolutely convergent:
\[
    \sum_{n=0}^{\infty} x^n = \frac{1}{1 - x}.
\]

\medskip

\textbf{Proposition 7.3.4} (Cauchy criterion) Let $(a_n)_{n=1}^{\infty}$ be a
decreasing sequence of non-negative real numbers (so $a_n \geq 0$ and
$a_{n+1} \leq a_n$ for all $n \geq 1$).
Then the series $\sum_{n=1}^{\infty} a_n$ is convergent if and only if the
series
\[
    \sum_{k=0}^{\infty} 2^k a_{2^k} = a_1 + 2a_2 + 4a_4 + 8a_8 + \dots
\]
is convergent.

\medskip

\textbf{Corollary 7.3.7} Let $q > 0$ be a real number. Then the series
$\sum_{n=1}^{\infty} \frac{1}{n^q}$ is convergent when $q > 1$ and divergent
when $q \leq 1$.

\medskip

\textbf{Proposition 7.4.3} (Rearrangement of series) Let $\sum_{n=0}^{\infty} a_n$
be an absolutely convergent series of real numbers, and let $f: \mathbb{N} \to \mathbb{N}$
be a bijection.
Then $\sum_{m=0}^{\infty} a_{f(m)}$ is also absolutely convergent, and has the
same sum:
\[
    \sum_{n=0}^{\infty} a_n = \sum_{m=0}^{\infty} a_{f(m)}.
\]

\medskip


\textbf{Theorem 7.5.1} (Root test) Let $\sum_{n=m}^{\infty} a_n$ be a series of real
numbers, and let $\alpha := \limsup_{n \to \infty} |a_n|^{1/n}$.

\textbf{(a)} If $\alpha < 1$, then the series $\sum_{n=m}^{\infty} a_n$ is absolutely
convergent (and hence convergent).

\textbf{(b)} If $\alpha > 1$, then the series $\sum_{n=m}^{\infty} a_n$ is not
convergent (and hence cannot be absolutely convergent either).

\textbf{(c)} If $\alpha = 1$, we cannot assert any conclusion.


\medskip


\textbf{Corollary 7.5.3} (Ratio test) Let $\sum_{n=m}^{\infty} a_n$ be a series of
non-zero numbers. (The non-zero hypothesis is required so that the ratios
$\frac{|a_{n+1}|}{|a_n|}$ appearing below are well-defined.)

\begin{itemize}
    \item If $\limsup_{n \to \infty} \frac{|a_{n+1}|}{|a_n|} < 1$, then the series
          $\sum_{n=m}^{\infty} a_n$ is absolutely convergent (hence convergent).

    \item If $\liminf_{n \to \infty} \frac{|a_{n+1}|}{|a_n|} > 1$, then the series
          $\sum_{n=m}^{\infty} a_n$ is not convergent (and thus cannot be absolutely
          convergent).

    \item In the remaining cases, we cannot assert any conclusion.
\end{itemize}


\medskip

\textbf{Proposition 7.5.4} We have
\(
\lim_{n \to \infty} n^{1/n} = 1.
\)

\end{document}