\documentclass{article}
\usepackage{xeCJK} % For CJK support

\usepackage[margin=1in]{geometry} % Set all margins to 1 inch
\usepackage{parskip}% Add more space

\usepackage{amsmath} % is needed to use the math environments align, aligned, gather, gathered, multline etc.
\usepackage{amsthm} % For theorem environments

\usepackage{braket} % For \set
\usepackage{soul} %For strikethough text

\usepackage[hidelinks]{hyperref}

\newcommand{\doubleplus}{\mathbin{{+}{+}}}

\begin{document}
\section{Page 20}
Notice that we can prove easily, using Axioms 2.1, 2.2, and induction (Axiom
2.5), that the sum of two natural numbers is again a natural number (why?).

\begin{quotation}

    \textbf{Axiom 2.1} $0$ is a natural number.

    \textbf{Axiom 2.2} If $n$ is a natural number, then $n\doubleplus$ is also a natural number.

    \textbf{Axiom 2.5} (Principle of mathematical induction). Let $P(n)$ be any property per
    taining to a natural number $n$. Suppose that $P(0)$ is true, and suppose that whenever
    $P(n)$ is true, $P(n\doubleplus)$ is also true. Then $P(n)$ is true for every natural number $n$.

    \textbf{Definition 2.2.1} (Addition of natural numbers). Let m be a natural number. To add
    zero to $m$, we define $0 +m := m$.
    Now suppose inductively that we have defined how to add $n$ to $m$.
    Then we can add $n\doubleplus$ to $m$ by defining $(n\doubleplus)+m := (n +m)\doubleplus$.

\end{quotation}

\begin{proof}
    Fix an any natural number $m$,
    we want to show that for any natural number $n$, $n + m$ is a natural number.

    \textbf{Base case}:  We have $0 + m = m$, which is a natural number by Axiom 2.1.

    \textbf{Inductive step}: Assume that for some natural number $n$, $n + m$ is a natural number.

    By Definition 2.2.1, we have $(n\doubleplus) + m = (n + m) \doubleplus$.
    Since $n + m$ is a natural number, by Axiom 2.2, $(n + m) \doubleplus$ is also a natural number.
    Thus, $(n\doubleplus) + m$ is a natural number.

    Therefore, by the principle of mathematical induction (Axiom 2.5),
    we conclude that for any natural number $n$, $n+m$ is again a natural number.

    Since the arbitrary choice of $m$ was made,
    we can conclude that the sum of two natural numbers is again a natural number.

\end{proof}

\section{Page 21}
As a particular corollary of Lemma 2.2.2 and Lemma 2.2.3 we see that $n\doubleplus =n +1$(why?).

\begin{quotation}

    \textbf{Lemma 2.2.2} For any natural number $n$, $n +0 = n$.

    \textbf{Lemma 2.2.3} For any natural numbers $n$ and $m$, $n +(m\doubleplus) = (n +m)\doubleplus$.

\end{quotation}

\begin{proof}
    In Lemma 2.2.3, let $m = 0$.
    Then we have
    \[n + (0\doubleplus) = (n + 0) \doubleplus\]
    By Lemma 2.2.2, we have $n + 0 = n$.
    Thus, the left hand side becomes $n + 1$.
    The right hand side becomes $(n + 0) \doubleplus = n \doubleplus$.
\end{proof}

\section{Page 23}
When $a = 0$ we have $0\le b$ for all $b$ (why?)

If $a > b$, then $a\doubleplus  > b$ (why?).

If $a = b$, then $a\doubleplus  > b$ (why?).

\begin{proof}
    See solution to Exercise 2.2.4
\end{proof}

\section{Another handy lemma in textbook Chapter 2.2}
\textbf{Lemma}. For any natural number $a$ and $b$, $a  <b\doubleplus$ iff. $a \le b$.
\begin{proof}
    Included in solution ot Exercise 2.2.5
\end{proof}

\section{Exercise 2.2.7}
Solution found from \url{https://math.stackexchange.com/a/4730660}

\textbf{Exercise 2.2.7} Let $n$ be a natural number,
and let $P(m)$ be a property pertaining to the natural numbers such that
whenever $P(m)$ is true, $P(m\doubleplus)$ is true.
Show that if $P(n)$ is true, then $P(m)$ is true for all $m \ge n$.
(This principle is sometimes referred to as the principle of induction starting from the base case n.)
\begin{quotation}
    \textbf{Definition 2.2.11} (Ordering of the natural numbers)
    Let $n$ and $m$ be natural numbers.
    We say that $n$ is greater than or equal to $m$, and write $n \ge m$ or $m \le n$,iff we
    have $n = m +a$ for some natural number $a$. We say that $n$ is strictly greater than $m$,
    and write $n > m$ or $m < n$,iff $n \ge m$ and $n= m$.

\end{quotation}

\begin{proof}
    Define $Q(m):=P(n+m)$. Note that, in particular, $Q(0)=P(n)$.

    We are looking to prove the statement "Whenever $P(m)$
    is true, $P(m\doubleplus )$ is true. If $P(n)$
    is true, then $P(m)$
    is true for $m\ge n$
    ", and we will do so via induction on $Q$.

    For the base case, we are already told we can assume $P(n)$
    is true, so $Q(0)$
    is true.

    Then for the inductive step,
    we have $Q(m)=P(n+m)\implies P((n+m)\doubleplus )=P(n+(m\doubleplus ))=Q(m\doubleplus )$
    (by  Lemma2.2.3, already quoted in above section not far away).
    But since $Q(m)\implies Q(m\doubleplus )$
    and $Q(0)$, then $Q(m)$ is true for every natural m
    (by Axiom 2.5 Principle of mathematical induction).

    Now, consider $P(m)$
    for $m\ge n$.
    By  Definition 2.2.11 (Ordering of the natural numbers), this means that $m=n+a$
    for some natural number $a$.
    That means we can write $P(m)=P(n+a)=Q(a)$,
    but by our inductive proof $Q(a)$ is true for any natural number $a$,
    therefore $P(m)$ is also true, and that completes the proof.
\end{proof}

\section{Exercise 3.2.1}
Exercise 3.2.1 Show that the universal specification axiom, Axiom 3.9, if assumed to be true,
would imply Axioms 3.3, 3.4, 3.5, 3.6,and 3.7. (If we assume that all natural numbers are objects,
we also obtain Axiom 3.8.) Thus, this axiom, if permitted, would simplify the foundations of set
theory tremendously (and can be viewed as one basis for an intuitive model of set theory known as
“naive set theory”). Unfortunately, as we have seen, Axiom 3.9 is “too good to be true”!
\begin{quotation}
    \textbf{Axiom 3.3} (Empty set).There exists a set $\emptyset$, known as the empty set,which contains
    no elements, i.e., for every object $x$ we have $x \notin \emptyset$.

    \textbf{Axiom 3.4} (Singleton sets \st{and pair sets}). If $a$ is an object, then there exists a set
    $\set{a}$ whose only element is $a$, i.e., for every object $y$,we have $y \in\set{a}$ if and only if
    $y =a$; we refer to $\set{a}$ as the singleton set whose element is $a$.

    \textbf{Axiom 3.5} (Pairwise union). Given any two sets $A$, $B$, there exists a set $A \cup B$,
    called the union of $A$ and $B$, which consists of all the elements which belong to $A$
    or $B$ or both. In other words, for any object $x$,
    \[x \in A\cup B \iff (x \in A\ \text{or}\ x \in B)\]

    \textbf{Axiom 3.6} (Axiom of specification). Let $A$ be a set, and for each $x \in  A$, let $P(x)$
    be a property pertaining to $x$ (i.e., for each $x \in  A$, $P(x)$ is either a true statement or
    a false statement). Then there exists a set, called $\set{x \in  A : P(x) is true}$ (or simply
    $\set{x \in  A : P(x)}$ for short), whose elements are precisely the elements $x$ in $A$ for
    which $P(x)$ is true. In other words, for any object $y$,
    \[y \in
        \set{x \in  A: P(x)\ \text{is true}}
        \iff(y \in  A\ \text{and}\ P(y)\ \text{is true})\]

    \textbf{Axiom 3.7} (Replacement). Let A be a set. For any object $x\in A$, and any object
    $y$, suppose we have a statement $P(x, y)$ pertaining to $x$ and $y$, such that for each
    $x\in A$ there is at most one $y$ for which $P(x,y)$ is true. Then there exists a set
    $\set{y : P(x, y)\text{ is true for some } x\in A}$, such that for any object $z$,
    \begin{gather*}
        z \in\set{y : P(x, y)\text{ is true for some }x\in A}\\
        \iff P(x,z)\text{ is true for some }x\in A
    \end{gather*}

    \textbf{Axiom 3.9} (Universal specification). (Dangerous!) Suppose for every object x we
    have a property $P(x)$ pertaining to $x$ (so that for every $x$, $P(x)$ is either a true
    statement or a false statement). Then there exists a set $\set{x : P(x)\ \text{is true}}$ such that
    for every object $y$,
    \[y \in\set{x : P(x)\ \text{is true}}\iff P(y)\ \text{is true}\]
\end{quotation}
\begin{proof}
    Axiom 3.3 (Empty set):
    Let $P(x)$ be the property that $x\neq x$.
    \[
        y\in\left\{x:x\neq x\ \textrm{is true}\right\}\iff y\neq y\ \textrm{is true}.
    \]

    Axiom 3.4 (Singleton sets \st{and pair sets}):
    Let $P(x)$ be the property that $x=a$.
    \[
        y\in\left\{x:x=a\ \textrm{is true}\right\}\iff y=a\ \textrm{is true}.
    \]

    Axiom 3.5 (Pairwise union):
    Let $P(x)$ be the property that $x\in A$ or $x\in B$.
    \[
        y\in\left\{x:x\in A\ \textrm{or}\ x\in B\ \textrm{is true}\right\}\iff y\in A\ \textrm{or}\ y\in B\ \textrm{is true}.
    \]

    Axiom 3.6 (Axiom of specification). Let $P(x)$ be the property that $x\in A$ and $P(x)$ is true.
    \[
        y\in\left\{x:x\in A\ \textrm{and}\ P(x)\ \textrm{is true}\right\}\iff y\in A\ \textrm{and}\ P(y)\ \textrm{is true}.
    \]

    Axiom 3.7 (Replacement) Let $P(y)$ be the property that there exists $x\in A$ such that $P(x,y)$ is true.
    \[
        z\in\left\{y:P(y)\ \textrm{is true}\right\}=\left\{y:P(x,y)\ \textrm{is true}\ \textrm{for some}\ x\in A\right\}\\
        \iff P(x,z)\ \textrm{is true}\ \textrm{for some}\ x.
    \]
\end{proof}

\section{Exercise 3.2.2}
Use the axiom of regularity (and the singleton set axiom) to show that if $A$ is a set, then $A\notin A$. Furthermore, show that if $A$ and $B$ are two sets, then either $A\notin B$ or $B\notin A$ (or both).
(One corollary of this exercise is worth noting: given any set $A$, there exists a mathematical object
that is not an element in $A$, namely $A$ itself. Thus one can always “add one more element” to a set
$A$ to create a larger set, namely $A \cup\set{A}$.)
\begin{quotation}

    \textbf{Axiom 3.10} (Regularity). If $A$ is a non-empty set, then there is at least one element
    $x$ of $A$ which is either not a set, or is disjoint from $A$.
\end{quotation}
\begin{proof}
    Suppose that $A$ is a non-empty set, and $A$ is an element of $A$.

    Let's consider set $\set{A}$, which can be constructed through Axiom 3.4:

    Obviously, $A\in\set{A}$

    By Axiom 3.10, $A$ is either not a set, or is disjoint from $\set{A}$, which leads to $A\cap\set{A}=\emptyset$.

    Since $A\in A$ and $A\in\set{A}$, this implies that $A\cap\set{A}=A$, a contradiction. While if $A$ is an empty set. By Axiom 3.3, $A\notin A$.

    Furthermore, suppose that $A,B$ are two sets and we both have $A\in B$ and $B\in A$. By Axiom 3.4 (and Axiom 3.5), we have $A,B\in\set{A,B}$. By Axiom 3.10, we have $A\cap\set{A,B}=\emptyset$ and $B\cap\set{A,B}=\emptyset$, but this is contradictive with $A\in B,B\in A$ and $A,B\in\set{A,B}$.
\end{proof}

\section{Exercise 3.2.3}
Show (assuming the other axioms of set theory) that the universal specification
axiom, Axiom 3.9, is equivalent to an axiom postulating the existence of a “universal set” $\Omega$
consisting of all objects (i.e., for all objects $x$,we have $x \in \Omega$). In other words, if Axiom 3.9 is true,
then a universal set exists, and conversely, if a universal set exists, then Axiom 3.9 is true. (This
helps explain why Axiom 3.9 is called the axiom of universal specification.) Note that if a universal
set
existed, then we would have $\Omega\in\Omega$ by Axiom 3.1, contradicting Exercise 3.2.2. Thus the
axiom of foundation specifically rules out the axiom of universal specification.
\begin{proof}
    Suppose that Axiom 3.8 it true, then we have
    $y\in\{x:x\in\Omega\}$.
    Conversely, if a universal set $\Omega$ exists, by Axiom 3.6, we have
    $y\in\{x\in\Omega:P(x)\ \textrm{is true}\}\implies y\in\{x:P(x)\ \textrm{is true}\}$.
    Hence we obtain Axiom 3.8.
\end{proof}

\section{Thoughts on the set of all sets}
Let $U$ be the set of all sets. Does such $U$ exist?
(Note that $U$ is different from the universal set $\Omega$ in the Exercise 3.2.3,
where $\Omega$ is the set consisting everything.)
\begin{proof}
    The proof is clipped from \href{https://pansci.asia/archives/75290}{PanSci}.
    We slightly modify the example in Russell’s Paradox ($\set{x|x\notin x}$) to
    \[B:=\set{x\in U|x\notin x}\]
    If $B\in B$, the by the definition of $B$, there is
    \[B\in B\implies B\in V \text{ and }B\notin B\tag{1}\]
    If $B\notin B$, since $U$ contains every set, we have $B\in V$,
    so $B$ complies the definition of B, which means
    \[B\in U\text{ and }B\notin B\implies B\in B\tag{2}\]
    With (1) and (2), we get
    \[B\in B\iff B\in U\text{ and }B\notin B\]
    This made the claim above to be true only when it's vacuously true,
    which is that $B\in V$ is not true. Thus lead to contradiction.
\end{proof}

\section{Exercise 3.4.2 (iii)}
Solution found from \url{https://math.stackexchange.com/a/4710434}

\textbf{Exercise 3.4.2} Let $f : X \to Y$ be a function from one set $X$ to another set $Y$,let $S$ be a subset of
$X$,and let $U$ be a subset of $Y$.

(i) What, in general, can one say about $f^{-1}( f (S))$ and $S$?

(ii) What about $f( f^{-1}(U))$ and $U$?

(iii) What about $f^{-1}( f ( f^{-1}(U)))$ and $f^{-1}(U)$?

\begin{proof}
    For (i) and (ii), there's solution in answer book.

    For (iii), the anwser is $f^{-1}( f ( f^{-1}(U)))=f^{-1}(U)$

    From (i), we get $f^{-1}(U)\subseteq f^{-1}( f ( f^{-1}(U)))$. So we only need to prove
    $f^{-1}( f ( f^{-1}(U)))\subseteq f^{-1}(U)$

    For any $x\in f^{-1}( f ( f^{-1}(U)))$, by definition,
    we have $f(x)\in f ( f^{-1}(U))$,
    so there exists $z\in f^{-1}(U)$ such that $f(x)=f(z)$.
    Now, since $z\in f^{-1}(U)$ then $f(z)\in U$,
    hence $f(x)\in U$. Therefore, $x\in f^{-1}(U)$.
\end{proof}

\section{Exercise 3.4.6 (ii)}
\textbf{Exercise 3.4.6} (ii)
Conversely, show that Axiom 3.11 can be deduced the preceding axioms of set theory if one
accepts Lemma 3.4.10 as an axiom. (This may help explain why we refer to Axiom 3.11 as
the “power set axiom”.)

\begin{quotation}
    \textbf{Axiom 3.11} (Power set axiom).
    Let $X$ and $Y$ be sets.Then there exists a set,denoted
    $Y^X$, which consists of all the functions from $X$ to $Y$, thus
    \[f \in Y^X \iff (f \text{ is a function with domain }X\text{ and codomain }Y)\]

    \textbf{Lemma 3.4.10} Let $X$ be a set. Then the set
    \[\set{Y : Y \text{ is a subset of }X}\]
    is a set. That is to say, there exists a set $Z$ such that
    $Y \in Z\iff Y \subseteq X$
    for all objects $Y$
\end{quotation}
\begin{proof}
    For (i) there's solution in answer book.

    From \url{https://math.stackexchange.com/a/409659}

    Since $X$ and $Y$ exist, thus $A\times B$ exists, so $\mathcal{P}(A\times B)$
    exists (see exercise 3.5.1). Thus by axiom 3.6 (Axiom of specification), we have that
    \[\set{f\in \mathcal{P}(X\times Y):f\text{ is a function }X\to Y}\]
    exists. But this is just $Y^X$, by definition. So we're done.
\end{proof}

\section{Exercise 3.5.1}
 (i) Suppose we \textit{define} the ordered pair $(x, y)$
for any objects $x$ and $y$ by the formula $(x, y) := \{\{x\}, \{x, y\}\}$
(thus using several applications of Axiom 3.4).
Thus, for instance, $(1,2)$ is the set $\{\{1\}, \{1,2\}\}$,
$(2,1)$ is the set $\{\{2\}, \{2,1\}\}$,
and $(1,1)$ is the set $\{\{1\}\}$.
Show that such a definition (known as the \textit{Kuratowski definition} of an ordered pair)
indeed obeys the property (3.5).

(ii) Suppose we instead define an ordered pair using the alternate definition $(x, y) := \{x, \{x, y\}\}$.
Show that this definition (known as the \textit{short definition} of an ordered pair)
also verifies (3.5) and is thus also an acceptable definition of ordered pair.
(Warning: this is tricky; one needs the axiom of regularity, and in particular Exercise 3.2.2.)

(iii) Show that regardless of the definition of ordered pair,
the Cartesian product $X \times Y$ of any two sets $X, Y$ is again a set.
(Hint: first use the axiom of replacement to show that for any $x \in X$,
the set $\{(x, y) : y \in Y\}$ is a set, and then apply the axiom of union.)

\begin{quotation}
    Axiom 3.4 is quoted in section Exercise 3.2.1.
    And Axiom of regularity is quoted in section Exercise 3.2.2.
    They are not quoted here again.

    \textbf{Axiom 3.2} (Equality of sets).
    Two sets $A$ and $B$ are equal, $A = B$,
    iff every element of $A$ is an element of $B$ and vice versa.
    To put it another way, $A = B$ if and only if every element $x$ of $A$ belongs also to $B$,
    and every element $y$ of $B$ belongs also to $A$.

    \textbf{Axiom 3.12} (Union). Let $A$ be a set, all of whose elements are themselves sets.
    Then there exists a set $\bigcup A$
    whose elements are precisely those objects which are elements of the elements of $A$,
    thus for all objects $x$,
    \[
        x \in\bigcup A \iff (x \in S \text{ for some } S \in A).
    \]

    \textbf{Definition 3.5.1} (Ordered pair). If $x$ and $y$ are any objects (possibly equal),
    we define the \textit{ordered pair} $(x, y)$ to be a new object,
    consisting of $x$ as its first component and $y$ as its second component.
    Two ordered pairs $(x, y)$ and $(x', y')$ are considered equal
    if and only if both their components match, i.e.,
    \[
        (x, y) = (x', y') \iff (x = x' \text{ and } y = y')\tag*{(3.5)}
    \]
\end{quotation}

There is solution in other anwser book, but it doesn't use the hint.
The proof below utilizes the hint.

\begin{proof}
    (i) If $(x, y) = (x', y')$, then by Kuratowski definition we have
    \[\set{x}=\set{x'}\text{ and }\set{x,y}=\set{x',y'}\]
    or
    \[\set{x}=\set{x',y'}\text{ and }\set{x,y}=\set{x'}\]
    In the first case, we have $x=x'$ and whatever $\set{x,y}$, $\set{x',y'}$
    are singleton or pairwise set, we all have $y=y'$.
    For the second case, $\set{x,y}$ and $\set{x',y'}$ have to be singleton set,
    then we can easily see that $x=x'=y=y'$.

    Conversely, if $x=x'$ and $y=y'$, then by axiom 3.2, we can easily conclude $(x, y) = (x', y')$.

    (ii) if $(x, y) = (x', y')$, then by short definition we have
    \[\set{x,\set{x,y}}=\set{x',\set{x',y'}}\]
    We first show that $\set{x,\set{x,y}}$ or $\set{x',\set{x',y'}}$ can't be singleton set.
    If $\set{x,\set{x,y}}$ is a singleton set, then we have $x=\set{x,y}$.
    So $x$ is a set, by axiom of regularity, we have $x$ or $\set{x,y}$ is disjoint from $\set{x,\set{x,y}}$.
    If $x=\set{x,y}$, we have $\set{x,y}=\set{\set{x,y},y}$,
    but $\set{x,y}\in\set{\set{x,y},y}\cap\set{x,\set{x,y}}$ will lead to contradiction.

    Now there remains two cases:
    \begin{itemize}
        \item Case 1: $x=x'$ and $\set{x,y}=\set{x',y'}$.
        \item Case 2: $\set{x,y}=x'$ and $x=\set{x',y'}$.
    \end{itemize}
    If it's the second case, we have $x$ and $\set{x,y}$ are sets.
    Let's examine if $\set{x,{x,y}}$ complies axiom of regularity.

    Since $x=\set{x',y'}=\set{\set{x,y},y'}$, $\set{x,y}\in x\cap\set{x,\set{x,y}}$.
    Besides, $x\in\set{x,y}\cap\set{x,\set{x,y}}$.
    So for the second case, there's a contradiction.
    For the first case, it's easy to verify $x=x'$ and $y=y'$.

    (iii) We follow the hint. Fix $x$, for any $y\in Y$, we can construct a property $P(y,z)$ such that
    \begin{itemize}
        \item Definition 3.5.1: Let $P(y,z)$ be $z=(x,y)$.
        \item Kuratowski definition: Let $P(y,z)$ be $z=\{\{x\},\{x,y\}\}$.
        \item Short definition: Let $P(y,z)$ be $z=\{x,\{x,y\}\}$.
    \end{itemize}
    In short, for a fixed $x$, we have constructed $\set{(x,y):y\in Y}$ as a set.
    For every $x\in X$, we can construct such set.
    Then we union all of these sets together, we have $X\times Y$, which by Axiom of union, is a set
\end{proof}

\end{document}